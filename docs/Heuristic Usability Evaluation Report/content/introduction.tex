% !TEX root =  ../report.tex
\section{Introduction}

The objective of this evaluation is to improve the current design of our application \textit{talio}. We're doing this by investigating our current design, finding out where it can be improved and make the necessary changes. 

Talio will be an application to manage to manage your tasks. You have boards and on each board you can add lists with tasks. The idea being that you add tasks and move them between lists as you're working on them. One might have 3 lists: to-do, doing, and done. However, it's also possible to add a lot more lists. The application is supposed to be flexible to use and is both aimed at basic and advanced users.

The application does not have a finished product yet. Therefore the evaluation will be based upon a prototype consisting of images showing the design and visually describing features of the application.
The design we're evaluating is a digital prototype. It consists partially of screenshots of the actual implementation and partially of designs made in paint.NET\cite{paintNET}, an image editing program.

This prototype is only based on the basic requirements of the application. To get a good idea of what the features of TALIO should be, refer to the list of features below:

A user should be able to ...
\begin{itemize}
    \item connect to a talio server of choice
    \item disconnect from the server without restarting the app
    \item connect multiple clients to the same server at the same time
    \item see all the lists and their respective cards in one overview
        \item set and change the names of lists for better organization
    \item create new lists and delete old lists, so they can re-organize their card system
    \item create cards so they can add new tasks to their TODO lists
    \item change card details so they can change what they wanted to do
    \item Put cards in certain lists, so they can change the state of their cards (e.g. move it from the TODO list to the DONE list)
    \item to delete cards, so they can undo mistakes or clean up the board
    \item to change the order of cards to have an implicit order of priority
    \item to see changes others perform in real-time, so you can easily work on a board together
    \item drag and drop tasks with the mouse, so the
\end{itemize}
These features all come from the backlog we were provided.\cite{backlog}